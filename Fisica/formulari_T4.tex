\documentclass[12pt]{article}

\newif\ifland
%\landtrue
\landfalse

\ifland
\usepackage[landscape,margin=0.5in,top=0.5in,bottom=0.5in,a4paper]{geometry}
\else
\usepackage[margin=1in,a4paper]{geometry}
\fi

\usepackage{amsmath,multicol,titlesec,graphicx,listings,nicefrac}

\usepackage[utf8]{inputenc}
\titlespacing{\section}{0pt}{5pt}{0pt}
\titlespacing{\subsection}{0pt}{5pt}{0pt}
\titlespacing{\subsubsection}{0pt}{5pt}{0pt}
\setlength{\parindent}{0pt}
\pagenumbering{gobble}
\titleformat*{\section}{\large\bfseries}
\titleformat*{\subsection}{\bfseries}
\setlength{\columnseprule}{1pt}
\let\u\relax
\newcommand{\u}[1]{\underline{#1}}
\newcommand{\ci}{\textbullet\;}
 
\begin{document}
\ifland
\raggedright
\begin{multicols}{4}
\fi

\section{Ones harm\`oniques}

\u{Moviment simple}: $f\left( t \right) A\sin\left( 2\pi f\left( t-t_0 \right) \right), x = vt_x$. \\
\u{Longitud d'ona}: $\lambda = vT$. \\
\u{Moviment ondulatori}: $f\left( x,t \right) = A\sin\left( 2\pi f\left( t-t_x \right) \right) = A\sin \left[ 2\pi \left( \frac{t}{T} - \frac{x}{\lambda} \right) \right] = A\sin\left( \omega t-kx \right)$.
$\omega = \frac{2\pi}{T}, k = \frac{2\pi}{\lambda}$. \\
\u{Velocitat de fase}: $\frac{\omega}{k}$.
% TODO Problemes 1, 3

\section{Ones electromagn\`etiques}

\u{Velocitat}: $c \implies \lambda = cT$
\u{Permeabilitat el\`ectrica absoluta}: $\varepsilon = \frac{D}{E}$. \\
\u{Constat diel\`ectrica}: $\varepsilon_0 = 8.85\times 10^{-12}$ [$\nicefrac{F}{m}$]. \\
\u{Permeabilitat magn\`etica absoluta}: $\mu = \frac{B}{H}$. \\
\u{Contant mag\`etica}: $\mu_0 = 4\pi \times 10^{-7}$ Tm$\text{A}^{-1}$. \\
\ci $c = \frac{1}{\sqrt{\varepsilon_0\mu_0}}$. \\
\u{Equaci\'o ona camp el\`ectric}: $\vec{E}\left( x,t \right) = \vec{E}_0\sin\left( kx \pm \omega t= \delta \right)$. \\
\u{Equaci\'o ona camp magn\`etic}: $\vec{B}\left( x,t \right) = \vec{B}_0\sin\left( kx \pm \omega t= \delta \right)$. \\
$\vec{B} = \frac{1}{c}\left[ \vec{u} \times \vec{E} \right], \vec{E} = c \left[ \vec{B} \times \vec{u} \right]$. \\
\u{Apunt sobre el producte vectorial}: $\vec{c} = \left[ \vec{a} \times \vec{b} \right] \implies |c| = ab\sin\theta$ i $c \bot a,b$. \\
\u{Regla de la m\`a dreta}: polze = $\vec{c}$, \'index = $\vec{a}$, cor = $\vec{b}$. \\
\u{Densitat instant\`ania camp ele\`ectric}: $\eta\left( x,t \right) = \frac{1}{2}\varepsilon_0 E^2 + \frac{B^2}{2\mu_0} = \varepsilon_0 E^2 = \varepsilon_0 E_0^2\sin^2\left( kx \pm \omega t + \delta \right)$ [$\nicefrac{J}{m^3}$]. \\
\u{Densitat mitjana camp el\`ectric}: $\eta = <\eta\left( x,t \right)> = \frac{1}{2}\varepsilon_0 E_0^2 = \frac{B_0^2}{2\mu_0}$. \\
\u{Intensitat instant\`ania ona electromagn\`etica}: $I\left( x,t \right) = \frac{P}{S} = c\eta\left( x,t \right) = c\varepsilon_0 E_0^2\sin^2\left( kx \pm \omega t+ \delta \right) = \frac{P}{4\pi r^2}$. \\
\u{Intensitat mitjana}: $I = <I\left( x,t \right)> = \frac{1}{2}c\varepsilon_0 E_0^2 = \frac{c B_0^2}{2\mu_0}$. \\
% TODO Problemes 7,8,12
\u{Llei de Plank}: $I\left( \nu,T \right) = \frac{2h\nu^3}{c^2 (e^{\frac{h\nu}{kT}} - 1)}$.
% TODO Problema 10

\section{Polarizaci\'on}


\u{Camp el\`ectric ona plana}: $\vec{E} = \vec{E}_0 e^{i\left( \vec{k}\cdot\vec{r} \pm \omega\cdot t \right)}$.\\
\u{Descomposició de l'amplitud d'ona (suma d'un vector paral·lel al pla d'incidència i un perpendicular)}: $\vec{E}_0 = E_{0\parallel} \cdot e^{i\theta_{\parallel}}\cdot \vec{u_{\parallel}} + E_{0\perp} \cdot e^{i\theta_{\perp}} \cdot \vec{u_{\perp}} $.\\
\u{Segons el valor de la diferència de fase $·\theta = \theta_{\parallel} - \theta_{\perp}$}. Polaritazions: lineal, 0 o $\pi$; circular, $\frac{\pi}{2}$ o $\frac{3\pi}{2}$ ($E_{0\parallel} = E_{0\perp}$); i elíptica, la resta.\\
\u{Primera polaritzador del polaritzador lineal}: $\vec{E}_{out} = \vec{E}_{\parallel}$. Ona d'entrada: $\vec{E}_{in} = E_{\parallel} \cdot \vec{u_{\parallel}} + E_{\perp}\cdot \vec{u_{\perp}}$. Si $E_{\parallel} = E_{\perp}$, amplitud entrada: $E_{0} = \sqrt{E_{\parallel}^{2} + E_{\perp}^{2}} = \sqrt{2}E_{\parallel}$. L'amplitud disminuex de factor $\sqrt{2}$. L'energia $\eta = \frac{1}{2} \varepsilon_0 E_{0}^{2}$ dism. de factor 1/2.\\
\u{Segon polaritzador del polaritzador lineal}: $\vec{E}_{out} = \vec{E}_{\parallel}$. $\vec{E}_{in} = E_{0}\cos(\alpha) \cdot \vec{u_{\parallel}} + E_{0}\sin(\alpha)\cdot \vec{u_{\perp}}$ on $\alpha$ és l'angle entre eix del polaritzador i eix de polarització de la llum incident. L'amplitud de camp elèctric dism. de factor $\left|\frac{\vec{E}_{out}}{\vec{E}_{in}}\right| = \cos(\alpha)$. L'energia del camp dism. de factor $\frac{\eta_{out}}{\eta_{in}} = \cos^2(\alpha)$.
% TODO Diapos 85-95
\section{Reflecci\'o i refracci\'o}

\u{\'Index de refracci\'o}: $n = \frac{V}{c}$, $V =$ velocitat de fase al medi. \\
\u{Medis homogenis i is\`otrops}: $n$ constant. \\
\u{Medis anis\`otrop}: $n$ en funci\'o de direcci\'o. \\
\u{Medis heterogenis}: $n$ varia en funci\'o del punt. \\
\u{Al canviar de medi}: $f$ const., $V$ i $\lambda$ varien. \\
% TODO Problema 17
\u{Llei de Snell (refracci\'o)}: $n_1\sin\theta_1 = n_2\sin\theta_2$. \\
\u{$\theta_i$}: Angle raig-normal a la superf\'icie. \\
\u{\`Angle cr\'itic o l\'imit}: $\theta_{\text{c}} = \text{arcsin} \frac{n_2}{n_1}, \, (n_2 < n_1).$
% TODO Problema 21

\section{Interfer\`encies}

%TODO Diapo 116-125

\section{L\`asers}

\u{Energia d'un fot\'o}: $E = hf = h\frac{c}{\lambda} = \hbar\omega, \, h = 6.626\cdot 10^{-34}$ [J$\cdot$s], $\hbar = \frac{h}{2\pi}$. \\
\u{Vector d'ona}: $\vec{k}, \, k = \|\vec{k}\| = \frac{2\pi}{\lambda}$. \\
\u{Moment lineal fot\'o}: $\vec{p} = \hbar\vec{k},\, p = \hbar k = \frac{hf}{c} = \frac{h}{\lambda}$. \\
\u{Energia de $N$ fotons}: $E = Nhf$. \\

\ifland
\end{multicols}
\fi
\end{document}
